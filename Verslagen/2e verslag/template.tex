\documentclass[english]{sareport}
% use the option peerreview for creating an anonymized version of your report
% E.g., \documentclass[english,peerreview]{sareport}

\usepackage[colorlinks, linkcolor=black, citecolor=black, urlcolor=black]{hyperref}


% Set all authors, if your group counts 2, set third author empty \authorthree{}
% Set the groupname as well
\authorone{Robin Haveneers (r0450702)}
\authortwo{Stef Verreydt (r0456110)}
\authorthree{Axel Lemmens (r0462440)}
\groupname{Haveneers-Verreydt-Lemmens}

\academicyear{2016--2017}

\casename{Shared Internet Of Things Infrastructure Platform}
\phasenumber{2a}
\phasename{ADD Application}


\begin{document}
\maketitle

\tableofcontents

\chapter{Introduction}\label{sec:introduction}
\todoinline{An introduction is not required, but if you include it, it should not be empty.}

\chapter{Attribute-driven design documentation}\label{sec:add}
\section{Decomposition 1: SIOTIP (M1, U2, UC6, UC8, UC13)}
\subsection{Module to decompose}
In this run we decompose \texttt{SIoTIP}.

\subsection{Selected architectural drivers}
The non-functional drivers for this decomposition are:

\begin{itemize}
	\item \emph{M1}: Integrate new sensor or actuator manufacturer
	\item \emph{U2}: Easy installation
\end{itemize}

The related functional drivers are:

\begin{itemize}
	\item \emph{UC6}: Insert a pluggable device into a mote
	\item \emph{UC8}: Initialise a pluggable device
	\item \emph{UC13}: Configure a pluggable device
\end{itemize}

\paragraph{Rationale}
\todo{M1 en U2 zijn beide high priority + ze gaan beide over het installeren van nieuwe sensoren. U2 gaat meer over de gebruiksvriendelijkheid, terwijl M1 echt gaat over een nieuwe sensor van een andere fabrikant. M1 wordt zeer moeilijk zonder U2.}

\subsection{Architectural design}
\paragraph{Topic}
Discussion of the solution selected for (a part of) one of the architectural
drivers.

\subsubsection*{Alternatives considered}
\paragraph{Alternatives for solution}
A discussion of the alternative solutions and why that were not selected.

\subsection{Instantiation and allocation of functionality}
\paragraph{Decomposition}
Main aspects of the resulting decomposition.

\subparagraph{ModuleB}
Per introduced component a paragraph describing its responsibilities.

\subparagraph{ModuleC}
Per introduced component a paragraph describing its responsibilities.

\begin{figure}[!htp]
	\centering
	%\includegraphics[width=0.8\textwidth]{}
	\missingfigure[figwidth=0.8\textwidth]{Component-and-connector diagram}
	\caption{Component-and-connector diagram of this decomposition.
	}\label{fig:it1-cc_main}
\end{figure}

\paragraph{Behaviour}
If needed and explanation of the behaviour of certain aspects of the design so
far.

\begin{figure}[!htp]
	\centering
	%\includegraphics[width=0.8\textwidth]{}
	\missingfigure[figwidth=0.8\textwidth]{Sequence diagram}
	\caption{Sequence diagram illustrating a key behavioural aspect.
	}\label{fig:it1-seq_aspect1}
\end{figure}

\paragraph{Deployment}
Rationale of the allocation of components to physical nodes.

\begin{figure}[!htp]
	\centering
	%\includegraphics[width=0.8\textwidth]{}
	\missingfigure[figwidth=0.8\textwidth]{Deployment diagram}
	\caption{Deployment diagram of this decomposition.
	}\label{fig:it1-depl_main}
\end{figure}

\subsection{Interfaces for child modules}
\subsubsection*{ModuleB}
\begin{itemize}
	\item InterfaceA
	\begin{itemize}
		\item \texttt{returnType operation1(ParamType param1)} throws TypeOfException
		\begin{itemize}
			\item Effect: Describe the effect of calling this operation.
			\item Exceptions:
			\begin{itemize}
				\item TypeOfException: Describe when this exception is thrown.
			\end{itemize}
		\end{itemize}

		\item \texttt{returnType operation2()}
		\begin{itemize}
			\item Effect: Describe the effect of calling this operation.
			\item Exceptions: None
		\end{itemize}
	\end{itemize}
\end{itemize}

\subsection{Data type definitions}
Describe per complex data type used in the interfaces what it represents.

\paragraph{returnType} This data element represents X.

\paragraph{ParamType} This data element represents Y.

\subsection{Verify and refine}
This section describes per component which (parts of) the remaining
requirements it is responsible for.

\paragraph{ModuleB}
\begin{itemize}
	\item \emph{Z1}: name
	\item \emph{UCd}: name
\end{itemize}

\paragraph{ModuleC}
\begin{itemize}
	\item \emph{UCba}: name\\Description which part of the original use case is
	the responsibility of this component.
\end{itemize}

\section{Decomposition 2: Module (drivers)}
\subsection{Module to decompose}
\subsection{Selected architectural drivers}
\subsection{Architectural design}
\subsection{Instantiation and allocation of functionality}
\subsection{Interfaces for child modules}
\subsection{Data type definitions}
\subsection{Verify and refine}

\chapter{Resulting partial architecture}\label{sec:architecture}
This section provides an over of the architecture constructed through ADD\@.

\section{Context diagram}
This subsection discusses the context diagram.

\begin{figure}[!htp]
	\centering
	%\includegraphics[width=0.8\textwidth]{}
	\missingfigure[figwidth=0.8\textwidth]{Context diagram for component-and-
		connector view.}
	\caption{Context diagram for the component-and-connector view.
	}\label{fig:cc_context}
\end{figure}

\section{Component-and-connector view}
A short discussion of the component-and-connector view with the key
decompositions if any.

\begin{figure}[!htp]
	\centering
	%\includegraphics[width=0.8\textwidth]{}
	\missingfigure[figwidth=0.8\textwidth]{Component-and-connector diagram}
	\caption{Primary diagram for the component-and-connector view.
	}\label{fig:cc_main}
\end{figure}

\begin{figure}[!htp]
	\centering
	%\includegraphics[width=0.8\textwidth]{}
	\missingfigure[figwidth=0.8\textwidth]{Key decomposition}
	\caption{Decomposition of a component shown in Figure~\ref{fig:cc_main}
	}\label{fig:decomp_decomp1}
\end{figure}

\section{Deployment view}
A short discussion of the allocation of components to physical nodes based on a
context diagram and a deployment diagram.

\begin{figure}[!htp]
	\centering
	%\includegraphics[width=0.8\textwidth]{}
	\missingfigure[figwidth=0.8\textwidth]{Context diagram for the allocation
		view.}
	\caption{Context diagram for the allocation view.}\label{fig:depl_context}
\end{figure}

\begin{figure}[!htp]
	\centering
	%\includegraphics[width=0.8\textwidth]{}
	\missingfigure[figwidth=0.8\textwidth]{Deployment diagram}
	\caption{Primary diagram for the allocation view.}\label{fig:depl_main}
\end{figure}


\end{document}
