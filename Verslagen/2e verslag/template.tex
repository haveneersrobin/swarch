\documentclass[english]{sareport}
% use the option peerreview for creating an anonymized version of your report
% E.g., \documentclass[english,peerreview]{sareport}

\usepackage[colorlinks, linkcolor=black, citecolor=black, urlcolor=black]{hyperref}


% Set all authors, if your group counts 2, set third author empty \authorthree{}
% Set the groupname as well
\authorone{Robin Haveneers (r0450702)}
\authortwo{Stef Verreydt (r0456110)}
\authorthree{Axel Lemmens (r0462440)}
\groupname{Haveneers-Verreydt-Lemmens}

\academicyear{2016--2017}

\casename{Shared Internet Of Things Infrastructure Platform}
\phasenumber{2a}
\phasename{ADD Application}


\begin{document}
\maketitle

\tableofcontents

\chapter{Introduction}\label{sec:introduction}
\todoinline{An introduction is not required, but if you include it, it should not be empty.}

\chapter{Attribute-driven design documentation}\label{sec:add}
\section{Decomposition 1: SIOTIP (M1, U2, UC6, UC8, UC13)}
\subsection{Module to decompose}
In this run we decompose \texttt{SIoTIP}.

\subsection{Selected architectural drivers}
The non-functional drivers for this decomposition are:

\begin{itemize}
	\item \emph{M1}: Integrate new sensor or actuator manufacturer
\end{itemize}

The related functional drivers are:

\begin{itemize}
	\item \emph{UC8}: Initialise a pluggable device
	\item \emph{UC9}: Configure pluggable device access rights
	\item \emph{UC10}: Consult and configure the topology
	\item \emph{UC13}: Configure pluggable device
\end{itemize}

\paragraph{Rationale}
M1 was chosen because it has high priority and because a lot of high priority usecases are linked to M1. \todo{Beter uitschrijven.}

\subsection{Architectural design}
\paragraph{Gateway for M1}
M1 states that new types of sensor and actuator data should be transmitted to the online service. To accomodate for this, we first created a class \textit{Gateway} (representing the versasense gateway), which is responsible for relaying raw the sensor/actuator data to our system. This way, when a new peripheral (sensor/actuator) connects, the gateway automatically start to transmit the raw data of the new peripheral to our system.
\paragraph{Intermediary for processing raw data for M1}
M1 also states that the new type of sensor data has to be processed by our system. To accomplish this we introduced a \texttt{peripheralDB} wich stores all possible peripheral types with their corresponding conversion. This \texttt{peripheralDB} offers an API to add new peripheral types. When a new peripheral type is introduced, it is the task of the manufracturer to add a new entry to the \texttt{peripheralDB} including a name and a conversionfunction. We also implemented a \texttt{DataProcessingManager} which is responsible for receiving peripheral data from the gateway, processing and converting it to internal structures (using the \texttt{peripheralDB}) and send it to the rest of our system. This increases semantic coherence by assigning only this specific task to the \texttt{DataProcessingManager}. By doing this, we ensure modifiability by making the \texttt{peripheralDb} the only component that needs to be changed in order to accomodate for new peripheral types.
\paragraph{Intermediary for distributing data for M1}
Another requirement in M1 states that the peripheral data should be sent to the applications of the assigned customer organisations and should be stored. To solve this, we made a \texttt{DataDB} and a \texttt{AssignmentDB}. To former is responsible the data-records of all peripherals while the latter maps peripherals to customer organisations. In addition to these new structures we indroduced a \texttt{DataManager} to consult the \texttt{AssignmentDB} and send any incomming data (from the \texttt{DataProcessingManager}) to all applications that are allowed to access this peripheral data. It is also the responsibility of the \texttt{DataManager} to store the sensor data into the \texttt{DataDB}.

\subsubsection*{Alternatives considered}
\paragraph{Alternatives for solution}
A discussion of the alternative solutions and why that were not selected.

\subsection{Instantiation and allocation of functionality}
\paragraph{Decomposition}
Main aspects of the resulting decomposition.

\subparagraph{ModuleB}
Per introduced component a paragraph describing its responsibilities.

\subparagraph{ModuleC}
Per introduced component a paragraph describing its responsibilities.

\begin{figure}[!htp]
	\centering
	%\includegraphics[width=0.8\textwidth]{}
	\missingfigure[figwidth=0.8\textwidth]{Component-and-connector diagram}
	\caption{Component-and-connector diagram of this decomposition.
	}\label{fig:it1-cc_main}
\end{figure}

\paragraph{Behaviour}
If needed and explanation of the behaviour of certain aspects of the design so
far.

\begin{figure}[!htp]
	\centering
	%\includegraphics[width=0.8\textwidth]{}
	\missingfigure[figwidth=0.8\textwidth]{Sequence diagram}
	\caption{Sequence diagram illustrating a key behavioural aspect.
	}\label{fig:it1-seq_aspect1}
\end{figure}

\paragraph{Deployment}
Rationale of the allocation of components to physical nodes.

\begin{figure}[!htp]
	\centering
	%\includegraphics[width=0.8\textwidth]{}
	\missingfigure[figwidth=0.8\textwidth]{Deployment diagram}
	\caption{Deployment diagram of this decomposition.
	}\label{fig:it1-depl_main}
\end{figure}

\subsection{Interfaces for child modules}
\subsubsection*{ModuleB}
\begin{itemize}
	\item InterfaceA
	\begin{itemize}
		\item \texttt{returnType operation1(ParamType param1)} throws TypeOfException
		\begin{itemize}
			\item Effect: Describe the effect of calling this operation.
			\item Exceptions:
			\begin{itemize}
				\item TypeOfException: Describe when this exception is thrown.
			\end{itemize}
		\end{itemize}

		\item \texttt{returnType operation2()}
		\begin{itemize}
			\item Effect: Describe the effect of calling this operation.
			\item Exceptions: None
		\end{itemize}
	\end{itemize}
\end{itemize}

\subsection{Data type definitions}
Describe per complex data type used in the interfaces what it represents.

\paragraph{returnType} This data element represents X.

\paragraph{ParamType} This data element represents Y.

\subsection{Verify and refine}
This section describes per component which (parts of) the remaining
requirements it is responsible for.

\paragraph{ModuleB}
\begin{itemize}
	\item \emph{Z1}: name
	\item \emph{UCd}: name
\end{itemize}

\paragraph{ApplicationManager}
\begin{itemize}
	\item \emph{Av2}: Application failure
\end{itemize}

\paragraph{DataDB}
\begin{itemize}
	\item \emph{P2}: Requests to the pluggable data database
\end{itemize}

\paragraph{Gateway}
\begin{itemize}
	\item \emph{Av3b}: Pluggable device/mote failure detection
	\item \emph{UC11a}: Relay data from pluggable devices to the system
\end{itemize}

\paragraph{DataProcessingManager}
\begin{itemize}
	\item \emph{UC11b}: Convert raw data to a usable form and send it to the DataManager
\end{itemize}

\paragraph{DataManager}
\begin{itemize}
	\item \emph{UC11c}: Send data to all ApplicationManagers who belong to Applications which are authorized to use the sensor from which the data is sent.
	\item \emph{UC24b}: Process request for historical data and consult DataDB to retrieve data.
\end{itemize}

\paragraph{ApplicationManager}
\begin{itemize}
	\item \emph{Av2}: Application failure
	\item \emph{Av3b}: Redundancy when pluggable devices/motes fail \todoinline{Of is dit voor Application zelf?}
	\item \emph{U2b}: Out of the box working when sensors are available
	\item \emph{UC11d}: Send the data received from the DataManager to the Application only if the Application has indicated it needs the data.
	\item \emph{UC25}: Access topology and available devices
\end{itemize}

\paragraph{Application}
\begin{itemize}
	\item \emph{U1}: Application updates
	\item \emph{UC24a}: Send request for historical data
\end{itemize}

\paragraph{Topology}
\begin{itemize}
	\item \emph{U2a}: Edit topology when installing a new mote
\end{itemize}

\paragraph{OtherFunctionality1}
\begin{itemize}
	\item \emph{Av1}: Communication between SIoTIP gateway and Online Service
	\item \emph{P1}: Large number of users
	\item \emph{M2}: Big data analytics on pluggable data and/or application usage data
	\item \emph{U2c}: Reintroducing motes \todoinline{Voor Topology?} and mandatory end-user assignment
	\item \emph{UC1}: Register a customer organisation
	\item \emph{UC2}: Register an end-user
	\item \emph{UC3}: Unregister an end-user
	\item \emph{UC4}: Install mote
	\item \emph{UC5}: Uninstall mote
	\item \emph{UC6}: Insert a pluggable device into a mote
	\item \emph{UC7}: Remove a pluggable device from its mote
	\item \emph{UC12}: Perform actuation command
	\item \emph{UC14}: Send heartbeat
	\item \emph{UC15}: Send notification
	\item \emph{UC16}: Consult notification message
	\item \emph{UC17}: Activate an application
	\item \emph{UC18}: Check and deactivate applications
	\item \emph{UC19}: Subscribe to application
	\item \emph{UC20}: Unsubscribe from application
	\item \emph{UC21}: Send invoice
	\item \emph{UC22}: Upload an application
	\item \emph{UC23}: Consult application statistics
	\item \emph{UC26}: Send application command or message to external front-end
	\item \emph{UC27}: Receive application command or message from external frontend
	\item \emph{UC28}: Log in
	\item \emph{UC29}: Log out
	\item \emph{
\end{itemize}

\section{Decomposition 2: Module (drivers)}
\subsection{Module to decompose}
\subsection{Selected architectural drivers}
\subsection{Architectural design}
\subsection{Instantiation and allocation of functionality}
\subsection{Interfaces for child modules}
\subsection{Data type definitions}
\subsection{Verify and refine}

\chapter{Resulting partial architecture}\label{sec:architecture}
This section provides an over of the architecture constructed through ADD\@.

\section{Context diagram}
This subsection discusses the context diagram.

\begin{figure}[!htp]
	\centering
	%\includegraphics[width=0.8\textwidth]{}
	\missingfigure[figwidth=0.8\textwidth]{Context diagram for component-and-
		connector view.}
	\caption{Context diagram for the component-and-connector view.
	}\label{fig:cc_context}
\end{figure}

\section{Component-and-connector view}
A short discussion of the component-and-connector view with the key
decompositions if any.

\begin{figure}[!htp]
	\centering
	%\includegraphics[width=0.8\textwidth]{}
	\missingfigure[figwidth=0.8\textwidth]{Component-and-connector diagram}
	\caption{Primary diagram for the component-and-connector view.
	}\label{fig:cc_main}
\end{figure}

\begin{figure}[!htp]
	\centering
	%\includegraphics[width=0.8\textwidth]{}
	\missingfigure[figwidth=0.8\textwidth]{Key decomposition}
	\caption{Decomposition of a component shown in Figure~\ref{fig:cc_main}
	}\label{fig:decomp_decomp1}
\end{figure}

\section{Deployment view}
A short discussion of the allocation of components to physical nodes based on a
context diagram and a deployment diagram.

\begin{figure}[!htp]
	\centering
	%\includegraphics[width=0.8\textwidth]{}
	\missingfigure[figwidth=0.8\textwidth]{Context diagram for the allocation
		view.}
	\caption{Context diagram for the allocation view.}\label{fig:depl_context}
\end{figure}

\begin{figure}[!htp]
	\centering
	%\includegraphics[width=0.8\textwidth]{}
	\missingfigure[figwidth=0.8\textwidth]{Deployment diagram}
	\caption{Primary diagram for the allocation view.}\label{fig:depl_main}
\end{figure}


\end{document}
