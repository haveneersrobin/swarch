\documentclass[english]{sareport}
% use the option peerreview for creating an anonymized version of your report
% E.g., \documentclass[english,peerreview]{sareport}

\usepackage[colorlinks, linkcolor=black, citecolor=black, urlcolor=black]{hyperref}


% Set all authors, if your group counts 2, set third author empty \authorthree{}
% Set the groupname as well
\authorone{Robin Haveneers (r0450702)}
\authortwo{Stef Verreydt (r0456110)}
\authorthree{Axel Lemmens (r0462440)}
\groupname{Haveneers-Verreydt-Lemmens}

\academicyear{2016--2017}

\casename{Shared Internet Of Things Infrastructure Platform}
\phasenumber{1}
\phasename{Domain Analysis}


\begin{document}
\maketitle

\tableofcontents

\chapter{Domain analysis}\label{sec:domain}
\section{Domain models}
This section shows the domain model(s).

\begin{figure}[!htp]
    \centering
    %\includegraphics[width=0.8\textwidth]{}
    \missingfigure[figwidth=0.8\textwidth]{Domain model}
    \caption{The domain model for the system.}\label{fig:domain_model}
\end{figure}

\section{Domain constraints}
In this section we provide additional domain constraints.

\begin{itemize}
    \item A customer organization can only subscribe to an application that is already registered with the system.
    \item A end-user only receives notifications from applications that the customer organization he works for is subscribed to. He only receives these notifcations if he/she is an individual who should receive notifications as configured in the application by the previously mentioned customer organization.
    \item The Online Service can only communicate with peripherals via their gateway.
    \item The connectivity between the network manager and the attached devices is done via 6LoWPAN.
    \item Application developers only receive rejection notification about the applications they want to deploy.
    \item An application should be idle until all the necesarry hardware is available and if the hardware is available it should be activated automatically. For example, a fire application requires at least 2 specific sensors per room.
    \item If a component of the hardware fails, the system should automatically switch to a nearby, similar device if possible, based on the topology provided by the infrastructure owner.
    \item Only persons who are allowed to do so, can issue commands to an application.
    \item Gateways contain limited storage space. Whenever they're runnin low on space, the oldest information should be removed.
\end{itemize}

\section{Glossary}
In this section, we provide a glossary of the most important terminology used
in this analysis.

\begin{itemize}
	\item \textbf{API}
	\item \textbf{Accelerometer}
	\item \textbf{Actuator}
	\item \textbf{Air Pressure}
	\item \textbf{Air Temperature/humidity}
	\item \textbf{Application Notification}
	\item \textbf{Application Usage Information}
	\item \textbf{Application provider}
	\item \textbf{Application}
	\item \textbf{Building}
	\item \textbf{Buzzer}
	\item \textbf{Connection Message}
	\item \textbf{Customer SLA}
	\item \textbf{Customer organization}
	\item \textbf{Data Message}
	\item \textbf{Development environement}
	\item \textbf{Distance}
	\item \textbf{End user}
	\item \textbf{Gateway identifier}
	\item \textbf{Gateway}
	\item \textbf{Hardware}
	\item \textbf{Heartbeat Message}
	\item \textbf{Infrastructure owner}
	\item \textbf{Library}
	\item \textbf{Light}
	\item \textbf{Llimited peripheral history}
	\item \textbf{Message}
	\item \textbf{Microphone}
	\item \textbf{Mobile app}
	\item \textbf{Mote manufacturer}
	\item \textbf{Mote}
	\item \textbf{Notification}
	\item \textbf{O2 gas}
	\item \textbf{Online Service}
	\item \textbf{Online Service}
	\item \textbf{Passive IR}
	\item \textbf{Peripheral Data}
	\item \textbf{Peripheral History}
	\item \textbf{Peripheral type}
	\item \textbf{Peripheral}
	\item \textbf{Power socket}
	\item \textbf{Rejection Notification}
	\item \textbf{SIoTIP Corporation}
	\item \textbf{Sandbox environement}
	\item \textbf{Sensor}
	\item \textbf{Server provider}
	\item \textbf{Server}
	\item \textbf{Service SLA}
	\item \textbf{Service providers}
	\item \textbf{Synchronisation protocol}
	\item \textbf{System Administrator}
	\item \textbf{System Notification}
	\item \textbf{Telecom SLA}
	\item \textbf{Telecom operators}
	\item \textbf{Topology}
	\item \textbf{UI}
	\item \textbf{Web interface}
\end{itemize}

\chapter{Functional requirements}\label{sec:functional}
\section*{Use case model}

\begin{figure}[!htp]
    \centering
    %\includegraphics[width=0.8\textwidth]{}
    \missingfigure[figwidth=0.8\textwidth]{Use case model}
    \caption{Use case diagram for the system.}\label{fig:use_case_model}
\end{figure}

\section{Use case overview}\label{sec:uc_overview}
\paragraph{UC1: Log in}
The user wishing to use the system provides his credentials.
The system verifies the credentials and authenticates the user.
If the provides credentials were not correct, the system does not authenticate the user.

\paragraph{UC2: Log off}
The user indicates he wants to log off from the system.
The system logs him of.

\paragraph{UC3: Enroll as application provider}
The organization wishing to becoma an application provider contacts SIoTIP and they negotiate the contract. The organization reveices API documentation and rules of conduct from SIoTIP. When the negotiations are conducted, the organization is provided dashboard accounts for the individual developers employed by the organization.

\paragraph{UC4: Add application}
The user logs into the application provider dashboard and uploads the new application. SIoTIP initiates a number of automated tests and shows the progress on the user's application provider dashboard. If the application passes all tests, the user receives a notification and the application is made available to customer organizations for subscription. If the application does not pass al tests, a SIoTIP system administrator performs a secondary review and decides whether to accept or reject the application. In the latter case, the user is notified of the reason of rejection.

\paragraph{UC5: Update existing application}
The user logs into the application provider dashboard and uploads the updated application. The user also indicates whether to automatically update existing instances of the application or to require customer organizations to subscribe to the application. SIoTIP initiates a number of automated tests and shows the progress on the user's application provider dashboard. If the application passes all tests, the user receives a notification and the application is made available to customer organizations for subscription. If the application does not pass all tests, a SIoTIP system administrator performs a secondary review and decides whether to accept or reject the application. In the latter case, the user is notified of the reason of rejection.

\paragraph{UC6: Register as new infrastructure owner}
The infrastructure owner contacts SIoTIP and negotiations are started. An infrastructure owner dashboard account is set up for the new user. The infrastructure owner provides the names of the currently renting companies, and SIoTIP contacts them for registration (See UC: Register new customer organization).

\paragraph{UC7: Register as new customer organization}
The new customer organization contacts or is contacted by SIoTIP and provides it's billing and contact information.

\paragraph{UC8: Subscribe to application}
The user logs into his dashboard, subscribes to the application and provides the needed information. The user is informed that the application will be activated once the required peripherals are installed. SIoTIP checks whether or not the customer organization has access to all the peripherals needed for the application. If not, the infrastructure owner is automatically notified of the subscription and the needed peripherals (see UC9: process peripheral request). Once all required hardware is installed, the application is activated and the user is notified.

\paragraph{UC9: Process peripheral request}
The infrastructure owner is notified of a new request for peripherals. SIoTIP automatically adds sufficient gateways needed to support these peripherals to the request, if any. The infrastructure owner approves or rejects the purchase of the hardware and a notification of the decision is sent to the user which requested the peripherals. If the infrastructure owner approved the request, the hardware is ordered from SIoTIP.

\paragraph{UC10: Install new hardware}
The infrastructure owner receives the hardware from SIoTIP and installs it. The infrastructure owner configures any new gateways to connect to the (WiFi?) local network. Once online, the gateways immediately connect to the Online Service to register themselves. The infrastructure owner then logs in to the infrastructure owner dashboard and provides the necessary topology information. Lastly, he assigns access rights to the customer organizations in his building to use the new peripherals.

\paragraph{UC11: Resolve hardware failure}
The system detects that a hardware component is no longer sending data and sends a notification to the infrastructure owner. The system also notifies all applications currently using the failing hardware component so that the applications can search for equivalent sensors in the topology. 

\paragraph{UC12: Configure end-users}
The user logs in to the customer organization dashboard and assigns other users to an application.
Nodig?

\paragraph{UC13: Transmit data}
The sensor sends data to the corresponding gateway. 


\section{Detailed use cases}
\subsection{\emph{UC1}: Log in}
\begin{itemize}
    \item \textbf{Name:} log in
    \item \textbf{Primary actor:} the User
    \item \textbf{Secondary actor(s)}: secondary actor(s)
    \item \textbf{Interested parties:} 
        \begin{itemize}
            \item \textit{System:} wants to authenticate its users.
        \end{itemize}

    \item \textbf{Preconditions:}
        \begin{itemize}
            \item The User is registered into the system and has credentials to prove his identity.
            \item Second precondition.
        \end{itemize}

    \item \textbf{Postconditions:}
        \begin{itemize}
            \item First postcondition.
            \item Second postcondition.
        \end{itemize}
        
    \item \textbf{Main scenario:} 
    \begin{enumerate}
       \item Step 1
       \item Step 2
       \item Step 3
       \item \ldots
    \end{enumerate}

    \item \textbf{Alternative scenarios:} 
    \begin{enumerate}
        \item [3b.] Alternative at step 3
    \end{enumerate}
    
    \item \textbf{Remarks:}
        \begin{itemize}
            \item First remark
        \end{itemize}
\end{itemize}

\paragraph{UC1: Name}
Short summary of this use case scenario

\section{Detailed use cases}
\subsection{\emph{UC1}: Log in}
\begin{itemize}
    \item \textbf{Name:} log in
    \item \textbf{Primary actor:} the User
    \item \textbf{Secondary actor(s)}: secondary actor(s)
    \item \textbf{Interested parties:} 
        \begin{itemize}
            \item \textit{System:} wants to authenticate its users.
        \end{itemize}

    \item \textbf{Preconditions:}
        \begin{itemize}
            \item First precondition.
            \item Second precondition.
        \end{itemize}

    \item \textbf{Postconditions:}
        \begin{itemize}
            \item First postcondition.
            \item Second postcondition.
        \end{itemize}
        
    \item \textbf{Main scenario:} 
    \begin{enumerate}
       \item Step 1
       \item Step 2
       \item Step 3
       \item \ldots
    \end{enumerate}

    \item \textbf{Alternative scenarios:} 
    \begin{enumerate}
        \item [3b.] Alternative at step 3
    \end{enumerate}
    
    \item \textbf{Remarks:}
        \begin{itemize}
            \item First remark
        \end{itemize}
\end{itemize}

\paragraph{UC1: Name}
Short summary of this use case scenario

\section{Detailed use cases}
\subsection{\emph{UC1}: Log in}
\begin{itemize}
    \item \textbf{Name:} log in
    \item \textbf{Primary actor:} the User
    \item \textbf{Secondary actor(s)}: secondary actor(s)
    \item \textbf{Interested parties:} 
        \begin{itemize}
            \item \textit{System:} wants to authenticate its users.
        \end{itemize}

    \item \textbf{Preconditions:}
        \begin{itemize}
            \item First precondition.
            \item Second precondition.
        \end{itemize}

    \item \textbf{Postconditions:}
        \begin{itemize}
            \item First postcondition.
            \item Second postcondition.
        \end{itemize}
        
    \item \textbf{Main scenario:} 
    \begin{enumerate}
       \item Step 1
       \item Step 2
       \item Step 3
       \item \ldots
    \end{enumerate}

    \item \textbf{Alternative scenarios:} 
    \begin{enumerate}
        \item [3b.] Alternative at step 3
    \end{enumerate}
    
    \item \textbf{Remarks:}
        \begin{itemize}
            \item First remark
        \end{itemize}
\end{itemize}

\chapter{Non-functional requirements}\label{sec:non-functional}
In this section, we model the non-functional requirements for the system in the
form of \emph{quality attribute scenarios}. We provide for each type
(availability, performance and modifiability) one requirement.

\section{Availability}
\subsection{\emph{Av1}: Name of the quality attribute scenario}
Shortly describe the context of the scenario.

\begin{itemize}
    \item \textbf{Source:} source
    \item \textbf{Stimulus:}
        \begin{itemize}
            \item Description of a first stimulus.
            \item Description of a second stimulus.
        \end{itemize}

    \item \textbf{Artifact:} the stimulated artifact
    \item \textbf{Environment:} the condition under which the stimulus occurs
    \item \textbf{Response:}
        \begin{itemize}
            \item Describe how the system should respond to the stimulus.
        \end{itemize}

    \item \textbf{Response measure:}
        \begin{itemize}
            \item Describe how the satisfaction of a response is measured.
        \end{itemize}
\end{itemize}

\paragraph{UC2: Name}
Short summary of this use case scenario

\section{Performance}
\subsection{\emph{P1}: Name of the quality attribute scenario}
Shortly describe the context of the scenario.

\begin{itemize}
    \item \textbf{Source:} source
    \item \textbf{Stimulus:}
        \begin{itemize}
            \item Description of a first stimulus.
            \item Description of a second stimulus.
        \end{itemize}

    \item \textbf{Artifact:} the stimulated artifact
    \item \textbf{Environment:} the condition under which the stimulus occurs
    \item \textbf{Response:}
        \begin{itemize}
            \item Describe how the system should respond to the stimulus.
        \end{itemize}

    \item \textbf{Response measure:}
        \begin{itemize}
            \item Describe how the satisfaction of a response is measured.
        \end{itemize}
\end{itemize}

\section{Modifiability}
\subsection{\emph{M1}: Name of the quality attribute scenario}
Shortly describe the context of the scenario.

\begin{itemize}
    \item \textbf{Source:} source
    \item \textbf{Stimulus:}
        \begin{itemize}
            \item Description of a first stimulus.
            \item Description of a second stimulus.
        \end{itemize}

    \item \textbf{Artifact:} the stimulated artifact
    \item \textbf{Environment:} the condition under which the stimulus occurs
    \item \textbf{Response:}
        \begin{itemize}
            \item Describe how the system should respond to the stimulus.
        \end{itemize}

    \item \textbf{Response measure:}
        \begin{itemize}
            \item Describe how the satisfaction of a response is measured.
        \end{itemize}
\end{itemize}

\section{Usability}
\subsection{\emph{U1}: Name of the quality attribute scenario}
Shortly describe the context of the scenario.

\begin{itemize}
    \item \textbf{Source:} source
    \item \textbf{Stimulus:}
        \begin{itemize}
            \item Description of a first stimulus.
            \item Description of a second stimulus.
        \end{itemize}

    \item \textbf{Artifact:} the stimulated artifact
    \item \textbf{Environment:} the condition under which the stimulus occurs
    \item \textbf{Response:}
        \begin{itemize}
            \item Describe how the system should respond to the stimulus.
        \end{itemize}

    \item \textbf{Response measure:}
        \begin{itemize}
            \item Describe how the satisfaction of a response is measured.
        \end{itemize}
\end{itemize}

\end{document}